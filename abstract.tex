We analyse the performance of a phonetic recogniser based on Context Dependent Deep Neural Networks (CD-DNNs) and Hidden Markov Models (HMMs) when the window of input features is not symmetric with respect to the current frame.
The objective is to reduce the latency of the system by reducing the number of future feature frames required to estimate the current output.

Our tests performed on the TIMIT database show that the performance does not degrade when the input window is shifted up to 5 frames in the past compared to common practice.
This corresponds to improving the latency by 50 ms in our settings.
Our tests also show that the best results (22.0\% PER) are not obtained with the symmetric window commonly employed, but with an asymmetric window with eight past and two future context frames.

The reduction in latency suggested by our results is critical for specific applications such as real-time lip synchronisation for telepresence, but may also be beneficial in general applications to improve the lag in human-machine spoken interaction.
\endinput

%%% Local Variables: 
%%% enable-local-variables: t
%%% ispell-local-dictionary: "british"
%%% mode: latex
%%% TeX-master: "dnnlatency"
%%% eval: (flyspell-mode)
%%% eval: (flyspell-buffer)
%%% End: 
